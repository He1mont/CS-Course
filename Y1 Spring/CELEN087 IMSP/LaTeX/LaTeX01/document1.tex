\documentclass[a4paper,11pt]{article}

\usepackage[margin=2cm]{geometry}
\setlength{\parindent}{0pt}
\setlength{\parskip}{9pt}
\usepackage{fancyhdr}
\pagestyle{fancy}

\usepackage{amsmath,amssymb} % for mathematical formulae

\title{\textsc{My First Document}}
\author{\textbf{\textit{\Large Aidin JALILZADEH}} \\ {\tiny 1234566}}
\date{\today}

\begin{document}
\maketitle

\section{Introduction}
Hello. \textbf{This is my first} \LaTeX document. I am very excited. Hello. This is my first \LaTeX document. I am very excited. Hello. This is my first \LaTeX document. I am very excited. Hello. This is my first \LaTeX document. I am very excited. Hello. This is my first \LaTeX document. \underline{I am very excited.}

\section{I forgot to say ...}
Hello. This is my first \LaTeX document. I am very excited. Hello. This is my first \LaTeX document. I am very excited. Hello. This is my first \LaTeX document. I am very excited. Hello. This is my first \LaTeX document. I am very excited. Hello. This is my first \LaTeX document. I am very excited. Hello. This is my first \LaTeX document. I am very excited. Hello. This is my first \LaTeX document. I am very excited. Hello. This is my first \LaTeX document. I am very excited. Hello. This is my first \LaTeX document. I am very excited. Hello. This is my first \LaTeX document. I am very excited.


\section{Next Steps}
Hello. This is my first \LaTeX document. I am very excited. Hello. This is my first \LaTeX document. I am very excited. Hello. This is my first \LaTeX document. I am very excited. Hello. This is my first \LaTeX document. I am very excited. Hello. This is my first \LaTeX document. I am very excited.

\subsection{Important}
Hello. This is my first \LaTeX document. I am very excited. Hello. This is my first \LaTeX document. I am very excited. Hello. This is my first \LaTeX document. I am very excited. Hello. This is my first \LaTeX document. I am very excited. Hello. This is my first \LaTeX document. I am very excited. Hello. This is my first \LaTeX document. I am very excited. Hello. This is my first \LaTeX document. I am very excited. Hello. This is my first \LaTeX document. I am very excited. Hello. This is my first \LaTeX document. I am very excited. Hello. This is my first \LaTeX document. I am very excited.

\subsubsection{Carefully Done!!}
Hello. This is my first \LaTeX document. I am very excited. Hello. This is my first \LaTeX document. I am very excited. Hello. This is my first \LaTeX document. I am very excited. Hello. This is my first \LaTeX document. I am very excited. Hello. This is my first \LaTeX document. I am very excited. Hello. This is my first \LaTeX document. I am very excited. Hello. This is my first \LaTeX document. I am very excited. Hello. This is my first \LaTeX document. I am very excited. Hello. This is my first \LaTeX document. I am very excited. Hello. This is my first \LaTeX document. I am very excited.


\section{Mathematics}

The equation of a circle centered at $(0,0)$ is given by $x^2+y^2=r^2$, where $r$ is the radius of the circle. Whereas the equation of a circle centered at $(\alpha,\beta)$ is given by $(x-\alpha)^2+(y-\beta)^2=r^2$.

The equation of an ellipse centred at $(0,0)$ is given by:
\begin{equation}
\frac{x^2}{a^2}+\frac{y^2}{b^2}=1 
\end{equation}

Here is a set of different equations:
\begin{eqnarray}
y &= & mx+b \nonumber \\
f(x)+g(x) &= & h(x) \nonumber \\
\sin(\pi - \alpha) &= & \sin \alpha \nonumber \\
x &= & r \cdot \cos(\delta)
\end{eqnarray}

Display equation:
\begin{equation}
e^{1-\sin(x)}
\end{equation}


























\end{document}